%%%
%%% local configuration adjustments
%%%

%%% You can change pre-defined colours here, override built-in macros from the
%%% style definition and standard library, as well as define macros needed by
%%% all local documents.

%%% e.g. adjust counterpoint (dark green) for data projectors where greens are
%%% far too bright, as well as green component of light colour and pure green
%%% (of course, it's a better solution to adjust the gamma settings of your monitor)
%%
%% \definecolor{counterpoint}{rgb}{.1, .3, 0}
%% \definecolor{light}{rgb}{.45, .3, .55}
%% \definecolor{puregreen}{rgb}{0, .35, 0}

%% ----- extra packages we need to load

\usepackage{tikz}
\usepackage{alltt}              % code examples with nicely formatted comments
\usepackage{hieroglf}           % hieroglyph font for the archeology example
\usepackage{rotating}
\usepackage{multirow}
\usepackage[normalem]{ulem}
\usepackage{calc}

\usetikzlibrary{arrows.meta}
\usetikzlibrary{shadows}
\usetikzlibrary{matrix}
\usetikzlibrary{positioning}

%% ----- predefined TikZ styles

%% diagram:
%%   box[=colour]   ... shaded text box with drop shadow
%%   arrow[=colour] ... thick arrow between boxes
%%   ghost          ... invisible box (white)
\tikzstyle{diagram box}[secondary]=[
rectangle, rounded corners=1ex, inner sep=5pt, minimum height=4ex, minimum width=4ex,
thick, draw=#1, top color=#1!5!white, bottom color=#1!25!white, drop shadow]

\tikzstyle{diagram dark box}[secondary]=[
rectangle, rounded corners=1ex, inner sep=5pt, minimum height=4ex, minimum width=4ex,
thick, draw=#1, text=white, top color=#1!70!white, bottom color=#1, drop shadow]

\tikzstyle{diagram ghost}=[
rectangle, rounded corners=1ex, inner sep=5pt, minimum height=4ex, minimum width=4ex,
thick, draw=white, color=white]

\tikzstyle{diagram arrow}[secondary]=[
->, >={Latex[width=4pt, length=4pt]}, ultra thick, color=#1]

%% matrix: **EXPERIMENTAL**
%%   fixed matrix[=grid]  ... matrix with fixed distances between cells
%%   cell matrix[=size]   ... cells with fixed size (as long as contents fit)
%%   bmatrix[=margin]     ... bracketed matrix with margin adjustment (margin = cell margin)
%%   nmatrix[=matrix]     ... plain matrix, but use same margin adjustment as bmatrix
%%   cellbox[=colour]     ... draw cell node as coloured box (must be applied as node style)
%%   boxnodes[=colour]    ... draw all cell nodes in cellbox style
%%   debug matrix         ... show node and matrix frames
\tikzstyle{fixed matrix}[1cm]=[
row sep={#1,between origins}, column sep={#1,between origins}, nodes={anchor=base}]
\tikzstyle{bmatrix}[4pt]=[
left delimiter={[}, right delimiter={]}, inner sep=1pt, outer xsep=-#1, nodes={inner sep=#1}]
\tikzstyle{nmatrix}[4pt]=[
inner sep=1pt, nodes={inner sep=#1}] % without delimiters, but adjusted spacing
\tikzstyle{cell matrix}[5mm]=[
nodes={minimum size=#1, text depth={0.3 * #1}, text height={0.7 * #1}, inner sep=0pt, anchor=center}]
\tikzstyle{cellbox}[primary]=[
fill=#1!40!white, draw=#1, line width=1pt]
\tikzstyle{boxnodes}[primary]=[
nodes={cellbox=#1}]
\tikzstyle{debug matrix}=[nodes={draw=red, thin}, draw=green, thin]

%% hacked latex dots that can be scaled properly, with some vertical adjustment
%% \scaleDots[<raise>]{<factor>}{<dots-command>}
%% \scaleVdots[<factor>]  ...  \vdots scaled by <factor>
%% \scaleDdots[<factor>]  ...  \ddots scaled by <factor>
%% \scaleCdots[<factor>]  ...  \cdots scaled by <factor>
\newcommand{\scaleDots}[3][0ex]{\scalebox{#2}{\raisebox{#1}{\normalsize\ensuremath{#3}}}}
\newcommand{\scaleVdots}[1][1]{\scaleDots[-0.5ex]{#1}{\vdots}}
\newcommand{\scaleCdots}[1][1]{\scaleDots{#1}{\cdots}}
\newcommand{\scaleDdots}[1][1]{\scaleDots[-0.3ex]{#1}{\ddots}}

%% ----- general copyright message (authors may change between versions of the tutorial)
\newcommand{\dsmcopyright}{%
  Copyright \textcopyright\ 2009--2019 Evert, Lenci, Baroni \& Lapesa | 
  Licensed under CC-by-sa version 3.0}


%% ----- automatically show TOC reminder at beginning of each subsection
\AtBeginSubsection[]
{
  \begin{frame}
    \frametitle{Outline}
    \tableofcontents[current,currentsubsection]
  \end{frame}
}

%% ----- some useful macros for the SIGIL course

%% \begin{Rcode} .. \end{Rcode}
\newenvironment{Rcode}[1][]{%
  \setbeamercolor{block title}{fg=counterpoint,bg=counterpoint!15!white}%
  \setbeamercolor{block body}{bg=counterpoint!5!white}\small%
  \begin{block}{#1}\begin{alltt}\ungap[1]}{%
      \ungap[1]\end{alltt}\end{block}} % \end{alltt} ... to deconfuse emacs

%% use \sbox{\Rbox} ... \usebox{\Rbox} to insert arbitray latex into Rcode environment
\newsavebox{\Rbox}

%% > plot(x,y)      \REM{this produces a scatterplot}
\newcommand{\REM}[2][\small]{\textsf{#1\color{primary}\# #2}}

%% nice colour for R output: \begin{Rout} .. \end{Rout}
\newenvironment{Rout}[1][\footnotesize]{%
  \begin{footnotesize}#1\color{secondary}\bfseries}{%
    \color{black}\mdseries\end{footnotesize}}

%% symbols for centroid vector and singular value matrix 
%% \newcommand{\vmu}[1][]{\boldsymbol{\mu}\ifthenelse{\equal{#1}{}}{}{^{(#1)}}}
\newcommand{\Msigma}{\boldsymbol{\Sigma}}

%% rotated column labels for table (to fit long text into narrow columns
\newcommand{\rotLabel}[2][60]{\begin{rotate}{#1}#2\end{rotate}}

%% allow very long matrices for row vectors
\setcounter{MaxMatrixCols}{20}
