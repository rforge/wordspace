\begin{frame}
  \frametitle{Why vector spaces?}
  %% \framesubtitle{}

  \begin{itemize}
    \item Vector spaces encode basic geometric intuitions
      \begin{itemize}
      \itemhand geometric interpretation of numerical feature lists
      \itemhand one reason why linear algebra is such a useful tool
      \end{itemize}
      \pause\gap
  \item Interpretation of vectors $\vx, \vy, \ldots \in \setR^n$ as
    \h{points} in $n$-dimensional Euclidean (= intuitive) space
    \begin{itemize}
    \item $n=2$ \so Euclidean plane
    \item $n=3$ \so three-dimensional Euclidean space
    \end{itemize}
    \pause\gap
  \item Exploit geometric intuition for analysis of DSM data\\
    as group of points or arrows in Euclidean space
    \begin{itemize}
    \item distance, length, direction, angle, dimension, \ldots
    \item intuitive in $\setR^2$ and $\setR^3$
    \item can be generalised to higher dimensions
    \itemhand I may refer to feature vectors for target terms as ``data points''
    \end{itemize}
  \end{itemize}
\end{frame}

\begin{frame}
  \frametitle{The geometric interpretation of vectors}
  \framesubtitle{Vectors as points}

  \begin{columns}[T]
    \begin{column}{48mm}
      \begin{itemize}
      \item Vectors like $\vu = (4,2)$ and $\vv = (3,5)$ can be understood as
        the coordinates of points in the Euclidean plane
      \item In this interpretation, vectors identify specific locations in the
        plane
      \end{itemize}
    \end{column}
    \begin{column}{55mm}
      \includegraphics[width=55mm]{img/1_vector_space_1}      
    \end{column}
  \end{columns}
\end{frame}

\begin{frame}
  \frametitle{The geometric interpretation of vectors}
  \framesubtitle{Vectors as arrows \& vector addition}

  \begin{columns}[T]
    \begin{column}{48mm}
      \begin{itemize}
      \item Vectors can also be interpreted as ``displacement~arrows'' between
        points
      \item Arrow from $\vu$ to $\vv$ is de-\\scribed by vector
        $(-1,3)$
      \item Calculated as pointwise difference between components of $\vv$ and
        $\vu$: $\vv - \vu = (v_1-u_1, v_2-u_2)$
      \item General operation: \h{vector addition}
      \end{itemize}
    \end{column}
    \begin{column}{55mm}
      \includegraphics[width=55mm]{img/1_vector_space_2}      
    \end{column}
  \end{columns}
\end{frame}

\begin{frame}
  \frametitle{The geometric interpretation of vectors}
  \framesubtitle{Vectors as arrows}

  \begin{columns}[T]
    \begin{column}{48mm}
      \begin{itemize}
      \item Vectors as arrows are position-independent
      \item $\vy - \vx = \vv - \vu$ if the
        relative positions of $\vx$ and $\vy$ are the same as
        those of $\vu$ and $\vv$
      \item Regardless of their location in the plane
      \end{itemize}
    \end{column}
    \begin{column}{55mm}
      \includegraphics[width=55mm]{img/1_vector_space_3}      
    \end{column}
  \end{columns}
\end{frame}

\begin{frame}
  \frametitle{The geometric interpretation of vectors}
  \framesubtitle{Direction \& scalar multiplication}

  \begin{columns}[T]
    \begin{column}{48mm}
      \begin{itemize}
      \item Intuitively, arrows have a length and direction
      \item Arrows point in the same direction iff they are multiples of each
        other: \h{scalar multiplication} $\lambda \vu = (\lambda u_1,
        \lambda u_2)$ with constant factor $\lambda\in \setR$
      \item For $\lambda < 0$, arrows have opposite directions
      \item $-\vu = (-1)\cdot \vu$ is the inverse arrow of $\vu$
      \end{itemize}
    \end{column}
    \begin{column}{55mm}
      \includegraphics[width=55mm]{img/1_vector_space_4}
    \end{column}
  \end{columns}
\end{frame}

\begin{frame}
  \frametitle{The geometric interpretation of vectors}
  \framesubtitle{Linking points and arrows}

  \begin{columns}[T]
    \begin{column}{48mm}
      \begin{itemize}
      \item Points in the plane can be identified by displace-\\ment arrows from
        fixed reference point
      \item A natural reference point is the \h{origin} $\vnull = (0,0)$
      \item These arrows are given by the same vectors as the point coordinates
      \end{itemize}
    \end{column}
    \begin{column}{55mm}
      \includegraphics[width=55mm]{img/1_vector_space_5}
    \end{column}
  \end{columns}
\end{frame}

%%% Local Variables: 
%%% mode: latex
%%% TeX-master: "../../workspace"
%%% End: 
