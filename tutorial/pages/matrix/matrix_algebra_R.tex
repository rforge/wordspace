\begin{frame}
  \frametitle{R as a toy DSM laboratory}
  %% \framesubtitle{}

  \begin{itemize}
  \item Matrix algebra is a powerful and convenient tool in numerical
    mathematics \so implement DSM with matrix operations
  \item Specialised (and highly optimised) libraries are available for various
    programming languages (C, C++, Perl, Python, \ldots)
  \item Some numerical programming environments are even based entirely on
    matrix algebra (Matlab, Octave, NumPy/Sage)
  \item Statistical software packages like \textbf{R} also support matrices
  \end{itemize}
  \pause
  \begin{columns}[c]
    \begin{column}{75mm}
      \begin{itemize}
      \item \textbf{R} as a \h{DSM laboratory} for toy models\\
        \url{http://www.r-project.org/}
      \item Integrates efficient matrix operations with statistical analysis,
        clustering, machine learning, visualisation, \ldots
      \end{itemize}
    \end{column}
    \begin{column}{20mm}
%%      \includegraphics[width=20mm]{img/Rlogo}
    \end{column}
  \end{columns}
\end{frame}

\begin{frame}[containsverbatim]
  \frametitle{Matrix algebra with R}
  %% \framesubtitle{}

  Vectors in R:
  \begin{itemize}
  \item \counterpoint{\texttt{u1 <- c(3, 0, 2)}}
  \item \counterpoint{\texttt{u2 <- c(0, 2, 2)}}
  \item \counterpoint{\texttt{v <- 1:6}}
  \item \counterpoint{\texttt{print(v)}}
    \begin{footnotesize}
\begin{verbatim}
[1] 1 2 3 4 5 6
\end{verbatim}
    \end{footnotesize}
  \end{itemize}
   
  \gap
  Defining matrices:
  \begin{itemize}
  \item \counterpoint{\texttt{A <- matrix(v, nrow=3)}}
  \item \counterpoint{\texttt{print(A)}}
    \begin{footnotesize}
\begin{verbatim}
     [,1] [,2]
[1,]    1    4
[2,]    2    5
[3,]    3    6
\end{verbatim}
    \end{footnotesize}
  \end{itemize}
\end{frame}

\begin{frame}[containsverbatim]
  \frametitle{Matrix algebra in R}
  %% \framesubtitle{}
  
  Matrix of column vectors:
  \begin{itemize}
  \item \counterpoint{\texttt{B <- cbind(u1, u2)}}
  \item \counterpoint{\texttt{print(B)}}
    \begin{footnotesize}
\begin{verbatim}
     u1 u2
[1,]  3  0
[2,]  0  2
[3,]  2  2
\end{verbatim}
    \end{footnotesize}
  \end{itemize}

  \gap
  Matrix of row vectors:
  \begin{itemize}
  \item \counterpoint{\texttt{C <- rbind(u1, u2)}}
  \item \counterpoint{\texttt{print(C)}}
    \begin{footnotesize}
\begin{verbatim}
   [,1] [,2] [,3]
u1    3    0    2
u2    0    2    2
\end{verbatim}
    \end{footnotesize}
  \end{itemize}
\end{frame}

\begin{frame}[containsverbatim]
  \frametitle{Matrix algebra in R}
  %% \framesubtitle{}
  
  Matrix multiplication:
  \begin{itemize}
  \item \counterpoint{\texttt{A \%*\% C}}
    \begin{footnotesize}
\begin{verbatim}
     [,1] [,2] [,3]
[1,]    3    8   10
[2,]    6   10   14
[3,]    9   12   18
\end{verbatim}
    \end{footnotesize}
  \item NB: \counterpoint{\texttt{*}} does \emph{not} perform matrix
    multiplication
  \end{itemize}

  \gap
  Also for multiplication of matrix with vector:
  \begin{itemize}
  \item \counterpoint{\texttt{C \%*\% c(1,1,0)}}
    \begin{footnotesize}
\begin{verbatim}
   [,1]
u1    3
u2    2
\end{verbatim}
    \end{footnotesize}
    \itemhand result of multiplication is a column vector (i.e.\ plain vectors
    are interpreted as column vectors in matrix operations)
  \end{itemize}
\end{frame}

\begin{frame}[containsverbatim]
  \frametitle{Matrix algebra in R}
  %% \framesubtitle{}
  
  Transpose of matrix:
  \begin{itemize}
  \item \counterpoint{\texttt{t(A)}}
    \begin{footnotesize}
\begin{verbatim}
     [,1] [,2] [,3]
[1,]    1    2    3
[2,]    4    5    6
\end{verbatim}
    \end{footnotesize}
  \end{itemize}

  \gap
  Transposition of vectors:
  \begin{itemize}
  \item \counterpoint{\texttt{t(u1)}} (row vector)
    \begin{footnotesize}
\begin{verbatim}
     [,1] [,2] [,3]
[1,]    3    0    2
\end{verbatim}
    \end{footnotesize}
    \gap
  \item \counterpoint{\texttt{t(t(u1))}} (explicit column vector)
    \begin{footnotesize}
\begin{verbatim}
     [,1]
[1,]    3
[2,]    0
[3,]    2
\end{verbatim}
    \end{footnotesize}
  \end{itemize}
\end{frame}

\begin{frame}[containsverbatim]
  \frametitle{Matrix algebra in R}
  %% \framesubtitle{}
  
  Rank of a matrix:
  \begin{itemize}
  \item \counterpoint{\texttt{qr(A)\$rank}}
    \begin{footnotesize}
\begin{verbatim}
2
\end{verbatim}
    \end{footnotesize}
  \item \counterpoint{\texttt{la.rank <- function (A) qr(A)\$rank}}
  \item \counterpoint{\texttt{la.rank(A)}}
  \end{itemize}
  
  \gap
  Column rank = row rank:
  \begin{itemize}
  \item \counterpoint{\texttt{la.rank(A) == la.rank(t(A))}}
    \begin{footnotesize}
\begin{verbatim}
[1] TRUE
\end{verbatim}
    \end{footnotesize}
  \end{itemize}

  \gap
  $\mathbf{A}^T\cdot \mathbf{A}$ is symmetric (can you prove this?):
  \begin{itemize}
  \item \counterpoint{\texttt{t(A) \%*\% A}}
  \end{itemize}
\end{frame}

%%% Local Variables: 
%%% mode: latex
%%% TeX-master: "../../workspace"
%%% End: 
