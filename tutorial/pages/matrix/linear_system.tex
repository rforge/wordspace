\begin{frame}
  \frametitle{Linear equation systems}
  %% \framesubtitle{}

  \begin{itemize}
  \item Recall that a linear system of equations can be written in compact
    matrix notation:%
    \only<beamer:1| handout:0>{
      \begin{align*}
        a_{11} x_1 + a_{12} x_2 + \dots + a_{1n} x_n &= b_1 \\
        a_{21} x_1 + a_{22} x_2 + \dots + a_{2n} x_n &= b_2 \\
        & \;\vdots \\
        a_{k1} x_1 + a_{k2} x_2 + \dots + a_{kn} x_n &= b_k \\
      \end{align*}}
    \only<beamer:2| handout:0>{
      \[
      \begin{bmatrix}
        a_{11} & \ldots & a_{1n} \\
        \vdots &        & \vdots \\
        a_{k1} & \ldots & a_{kn} 
      \end{bmatrix}
      \cdot
      \begin{bmatrix}
        x_1 \\ x_2 \\ \vdots \\ x_n
      \end{bmatrix}
      =
      \begin{bmatrix}
        b_1 \\ \vdots \\ b_k
      \end{bmatrix}
      \]}
    \only<beamer:3-| handout:1>{
      \[
      \mathbf{A} \cdot \vx = \vb
      \]}
  \item<4-> Obviously, $\mathbf{A}$ describes a linear map $f: \setR^n\to
    \setR^k$, and the linear system of equations can be written $f(\vx) = \vb$
  \item<5-> This linear system can be solved iff $\vb\in \Image{f}$, i.e.\ iff
    $\vb$ is a linear combination of the column vectors of $\mathbf{A}$
  \item<6-> The solution is given by the coefficients $x_1, \ldots, x_n$\\ of
    this linear combination
  \end{itemize}
  \addnote{Note that the linear system can be solved by the method of Gaussian
    elimination (which basically transforms $\mathbf{A}$ into a triangular matrix).}%
  \addnote{Gaussian elimination can also be used to compute the inverse
    $\mathbf{A}^{-1}$ of a regular matrix $\mathbf{A}$}% 
\end{frame}

\begin{frame}
  \frametitle{Linear equation systems}
  %% \framesubtitle{}

  \begin{itemize}
  \item The linear system has a solution for arbitrary $\vb\in \setR^k$\\
    iff $f$ is surjective, i.e.\ iff $\Rank{\mathbf{A}} = k$%
    \pause
  \item Solutions of the linear system are unique iff $f$ is injective, i.e.\
    iff $\Rank{\mathbf{A}} = n$ (the column vectors are linearly independent)%
    \pause
  \item If $k=n$ (i.e.\ $\mathbf{A}$ is a square matrix), the linear map $f$ is an
    endomorphism.  Consequently, the linear system has a unique solution for
    arbitrary $\vb$ iff $\det \mathbf{A} \neq 0$%
    \pause
  \item In this case, the solution can be computed with the inverse function
    $f^{-1}$ or the inverse matrix $\mathbf{A}^{-1}$:
    \[
    \vx = f^{-1}(\vb) = \mathbf{A}^{-1}\cdot \vb
    \]
    \ungap[1.5]
    \begin{itemize}
      \itemhand {\footnotesize practically, $\mathbf{A}^{-1}$ is often
        determined by solving the corresponding linear system of equations}
    \end{itemize}
  \end{itemize}
\end{frame}

\begin{frame}[fragile]
  \frametitle{Linear equation systems}
  %% \framesubtitle{}

  Solving equation systems in R:
  \begin{itemize}
  \item \counterpoint{\texttt{A <- rbind(c(1,3), c(2,-1))}}
  \item \counterpoint{\texttt{b <- c(5,3)}}
  \item \counterpoint{\texttt{la.rank(A)}} (test that $\mathbf{A}$ is invertible)%
  \item<2-> \counterpoint{\texttt{A.inv <- solve(A)}} (inverse matrix $\mathbf{A}^{-1}$)
  \item<2-> \counterpoint{\texttt{print(round(A.inv, digits=3))}}
    \begin{footnotesize}
\begin{verbatim}
      [,1]   [,2]
[1,] 0.143  0.429
[2,] 0.286 -0.143
\end{verbatim}
    \end{footnotesize}
    
  \item<3-> \counterpoint{\texttt{A.inv \%*\% b}}
    \begin{footnotesize}
\begin{verbatim}
     [,1]
[1,]    2
[2,]    1
\end{verbatim}
    \end{footnotesize}
  \item<3-> \counterpoint{\texttt{solve(A, b)}} (recommended: calculate
    $\mathbf{A}^{-1}\cdot \vb$ directly)
  \end{itemize}
  
\end{frame}

%%% Local Variables: 
%%% mode: latex
%%% TeX-master: "../../workspace"
%%% End: 
