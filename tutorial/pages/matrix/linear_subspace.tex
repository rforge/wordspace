\begin{frame}
  \frametitle{Linear combinations \& dimensionality}
  %% \framesubtitle{}

  \begin{itemize}
  \item \h{Linear combination} of vectors $\vu[1],\ldots, \vu[n]$:
    \[ \lambda_1 \vu[1] + \lambda_2 \vu[2] + \dots + \lambda_n \vu[n] \]
    for any coefficients $\lambda_1, \ldots, \lambda_n \in \setR$
    \begin{itemize}
    \item intuition: all vectors that can be constructed from $\vu[1],\ldots,
      \vu[n]$ using the basic vector operations
    \end{itemize}
    \pause\gap
  \item $\vu[1],\ldots, \vu[n]$ are \h{linearly independent} iff
    \[ 
    \lambda_1 \vu[1] + \lambda_2 \vu[2] + \dots + \lambda_n \vu[n] = \vnull 
    \]
    implies $\lambda_1 = \lambda_2 = \dots = \lambda_n = 0$%
    \pause\gap
  \item Otherwise, they are \hh{linearly dependent}
    \begin{itemize}
    \item equivalent: one $\vu[i]$ is a linear combination of the other vectors
    \end{itemize}
  \end{itemize}
\end{frame}

\begin{frame}
  \frametitle{Linear combinations \& dimensionality}
  %% \framesubtitle{}

  \begin{itemize}
  \item Largest $n\in \setN$ for which there is a set of $n$
    linearly independent vectors $\vu[i]\in V$ is called the \h{dimension}
    of $V$: $\dim V = n$ 
  \item It can be shown that $\dim \setR^n = n$%
    \pause\gap
  \item If there is no maximal number of linearly independent vectors, the
    vector space is \h{infinite-dimensional} ($\dim V = \infty$)
  \item An example is $\dim \mathcal{C}[a,b] = \infty$ (easy to show)%
    \pause\gap
  \item Every finite-dimensional vector space $V$ is
    \h{isomorphic} to the Euclidean space $\setR^n$ (with $n = \dim V$)
    \begin{itemize}
      \itemhand We will be able to prove this in a little while
    \end{itemize}
  \end{itemize}
\end{frame}

\begin{frame}
  \frametitle{Basis \& coordinates}
  %% \framesubtitle{}

  \begin{itemize}
  \item A set of vectors $\vb[1], \ldots, \vb[n] \in V$ is called a
    \h{basis} of $V$\\ iff every $\vu\in V$ can be written as a linear
    combination
    \[ 
    \vu = x_1 \vb[1] + x_2 \vb[2] + \dots + x_n \vb[n]
    \]
    with unique coefficients $x_1, \ldots, x_n$
  \item Number of vectors in a basis = $\dim V$%
    \pause\gap
  \item For every $n$-dimensional vector space $V$, a set of $n$ vectors
    $\vb[1], \ldots, \vb[n]\in V$ is a basis iff they are linearly independent
    \begin{itemize}
    \itemhand Can you think of a proof?
    \end{itemize}
  \end{itemize}
\end{frame}

\begin{frame}
  \frametitle{Basis \& coordinates}
  %% \framesubtitle{}

  \begin{itemize}
  \item The unique coefficients $x_1, \ldots, x_n$ are called the
    \h{coordinates} of $\vu$ wrt.\ the basis $B \coloneq \bigl( \vb[1], \ldots,
      \vb[n] \bigr)$:
    \[ 
    \vu \equiv_B
    \begin{bmatrix}
      x_1\\ x_2\\ \vdots\\ x_n
    \end{bmatrix}
    \eqcolon \vx
    \]
    \pause
  \item $\vx\in \setR^n$ is the \h{coordinate vector} of $\vu\in V$ wrt.\ $B$
    \begin{itemize}
      \itemhand $V$ is isomorphic to $\setR^n$ by virtue of this
      correspondence
    \item[]\pause
    \end{itemize}
  \item We can think of the rows (or columns) of a DSM matrix $\mathbf{M}$ as
    coordinates in an abstract vector space
    \begin{itemize}
    \item coordinate transformations play an important role for DSMs
    \end{itemize}
  \end{itemize}
\end{frame}

\begin{frame}
  \frametitle{Basis \& coordinates}
  %% \framesubtitle{}

  \begin{itemize}
  \item The components $(u_1, u_2, \ldots, u_n)$ of a number vector $\vu \in
    \setR^n$ correspond to its \h{natural coordinates}
    \[ 
    \vu = (u_1, u_2, \ldots, u_n)  \equiv_E
    \begin{bmatrix}
      u_1\\ u_2\\ \vdots\\ u_n
    \end{bmatrix}
    \]
    according to the \h{standard basis} $\ve[1], \ldots, \ve[n]$ of $\setR^n$:
    \begin{align*}
      \ve[1] &= (1, 0, \ldots, 0)\\
      \ve[2] &= (0, 1, \ldots, 0)\\
      & \;\vdots\\
      \ve[n] &= (0, 0, \ldots, 1)
    \end{align*}
  \end{itemize}
\end{frame}

\begin{frame}
  \frametitle{Basis \& coordinates}
  %% \framesubtitle{}

  \begin{columns}[T]
    \begin{column}{35mm}
      \begin{itemize}
      \item<1-> $\vu = (4,5)\in \setR^2$
      \item<1-> Basis $B$ of $\setR^2$:
        \ungap[.5]
        \begin{align*}
          \vb[1] &= (2, 1)\\
          \vb[2] &= (-1, 1)
        \end{align*}
        \ungap[2]
      \item<1-> $\vu \equiv_B
        \begin{bmatrix}
          3\\ 2
        \end{bmatrix}$
      \item<2-> Standard basis:
        \ungap[.5]
        \begin{align*}
          \ve[1] &= (1, 0)\\
          \ve[2] &= (0, 1)
        \end{align*}       
        \ungap[2]
      \item<2-> $\vu \equiv_E
        \begin{bmatrix}
          4\\ 5
        \end{bmatrix}$
      \end{itemize}
    \end{column}
    \begin{column}{65mm}
      \gap[2]
      \only<beamer:1| handout:0>{\includegraphics[width=65mm]{img/1_basis_1}}
      \only<beamer:2| handout:1>{\includegraphics[width=65mm]{img/1_basis_2}}
    \end{column}
  \end{columns}
\end{frame}

\begin{frame}
  \frametitle{Linear subspaces}
  %% \framesubtitle{}

  \begin{itemize}
  \item The set of all linear combinations of vectors $\vb[1], \ldots,
    \vb[k]\in V$ is called the \h{span}
    \[
    \Span{\vb[1], \ldots, \vb[k]} \coloneq
    \setdefscale{\lambda_1 \vb[1] + \dots + \lambda_k \vb[k]}{\lambda_i \in \setR}
    \]
    \pause\ungap[1.5]
  \item $\Span{\vb[1], \ldots, \vb[k]}$ forms a \h{linear subspace} of $V$
    \begin{itemize}
    \item a linear subspace is a subset of $V$ that is closed under vector
      addition and scalar multiplication
    \end{itemize}
    \pause
  \item $\vb[1], \ldots, \vb[k]$ form a basis of $\Span{\vb[1], \ldots,
      \vb[k]}$\\ iff they are linearly independent
  \itemhand {\small Can you prove that every linear subspace of $\setR^n$ has a basis?}%
  \pause
  \item The \h{rank} of vectors $\vb[1], \ldots, \vb[k]$ is the dimension of
    their span, corresponding to the largest number of linearly independent
    vectors among them
  \end{itemize}
\end{frame}

\begin{frame}
  \frametitle{Linear combinations \& linear subspace}

  \begin{itemize}
  \item Example: linear subspace $U\subseteq \setR^3$ spanned by vectors
    $\vb[1] = (6,0,2)$, $\vb[2] = (0,3,3)$ and $\vb[3] = (3,1,2)$
    \begin{itemize}
    \item $\dim U = 2$ 
      \only<beamer:1| handout:0>{(why?)}%
      \only<beamer:2| handout:1>{(because $\vb[2] = 3 \vb[3] - \frac{3}{2} \vb[1]$)}
    \end{itemize}
  \end{itemize}

  \begin{center}
    \includegraphics<beamer:1| handout:0>[width=5cm]{img/1_subspace_1}
    \includegraphics<beamer:2| handout:1>[width=5cm]{img/1_subspace_2}
  \end{center}
\end{frame}


%%% Local Variables: 
%%% mode: latex
%%% TeX-master: "../../workspace"
%%% End: 
