\begin{frame}
  \frametitle{Coordinate transformations}
  %% \framesubtitle{}

  \begin{itemize}
  \item We want to \h{transform} between coordinates with respect to\\ a basis
    $\vb[1],\ldots,\vb[n]$ and standard coordinates in $\setR^n$
  \end{itemize}

  \begin{center}
    \includegraphics[width=70mm]{img/1_basis_2}
  \end{center}
\end{frame}

\begin{frame}
  \frametitle{Coordinate transformations}
  %% \framesubtitle{}

  \begin{itemize}
  \item The basis can be represented by a matrix $\mathbf{B}$ whose columns are the
    standard coordinates of $\vb[1],\ldots,\vb[n]$
  \item Given a vector $\vu \in \setR^n$ with standard coordinates $\vu
    \equiv_E \vx$ and $\mathbf{B}$-coordinates $\vu \equiv_B \vy$, we have
    \[ \vu = y_1 \vb[1] + \dots + y_n \vb[n] \]
    \pause\ungap
  \item In standard coordinates, this equation corresponds to matrix
    multiplication:
    \[ \vx = \mathbf{B}\cdot \vy \]
  \item[\So] Matrix $\mathbf{B}$ transforms $B$-coordinates into standard coordinates
  \end{itemize}
\end{frame}

\begin{frame}
  \frametitle{Coordinate transformations}
  %% \framesubtitle{}

  \begin{itemize}
  \item  To transform from standard coordinates into $B$-coordinates,
    i.e.\ from $\vx$ to $\vy$, we must solve the linear system $\vx = \mathbf{B}
    \vy$%
    \pause
  \item Since the $\vb[i]$ are linearly independent, $\mathbf{B}$ is regular and the
    inverse $\mathbf{B}^{-1}$ exists, so that
    \[ \vy = \mathbf{B}^{-1} \vx \]
  \item[\So] The inverse matrix $\mathbf{B}^{-1}$ transforms from standard coordinates
    into $B$-coordinates%
    \pause
  \item Recall that $\mathbf{B} \mathbf{B}^{-1} = \mathbf{B}^{-1} \mathbf{B} =
    \mathbf{I}$ (transform back \& forth)
  \item Transformation from $B$-coordinates ($\vu \equiv_B \vy$) into
    arbitrary $C$-coordinates ($\vu \equiv_C \vz$):
    \[ \vz = \mathbf{C}^{-1} \mathbf{B} \vy \]
  \end{itemize}
\end{frame}

\begin{frame}[c]
  \frametitle{Coordinate transformations: an example}
  %% \framesubtitle{}

  \begin{center}
    \includegraphics[width=80mm]{img/1_basis_2}
  \end{center}
\end{frame}

\begin{frame}
  \frametitle{Coordinate transformations: an example}
  %% \framesubtitle{}

  \begin{itemize}
  \item Basis $\vb[1] = (2,1)$, $\vb[2] = (-1,1)$ with matrix representation
    \[ 
    \mathbf{B} = \begin{bmatrix}
      2 & -1 \\ 1 & 1
    \end{bmatrix},
    \quad
    \mathbf{B}^{-1} = \begin{bmatrix}
      \frac{1}{3} & \frac{1}{3} \\ \rule{0mm}{1.4em}-\frac{1}{3} & \frac{2}{3}
    \end{bmatrix}
    \]
    \pause
  \item Vector $\vu = (4,5)$ with standard and $\mathbf{B}$-coordinates
    \[
    \vu \equiv_E \begin{bmatrix}
      4 \\ 5
    \end{bmatrix},
    \quad
    \vu \equiv_C \begin{bmatrix}
      3 \\ 2
    \end{bmatrix}
    \]
    \pause
  \item Check that these equalities hold:
    \[
    \begin{bmatrix}
      4 \\ 5
    \end{bmatrix}
    =
    \begin{bmatrix}
      2 & -1 \\ 1 & 1
    \end{bmatrix}
    \begin{bmatrix}
      3 \\ 2
    \end{bmatrix},
    \quad
    \begin{bmatrix}
      3 \\ 2
    \end{bmatrix}
    =
    \begin{bmatrix}
      \frac{1}{3} & \frac{1}{3} \\ \rule{0mm}{1.4em}-\frac{1}{3} & \frac{2}{3}
    \end{bmatrix}  
    \begin{bmatrix}
      4 \\ 5
    \end{bmatrix}
    \]
    \pause
  \item Now perform the calculations in R!
  \end{itemize}
\end{frame}

%%% Local Variables: 
%%% mode: latex
%%% TeX-master: "../../workspace"
%%% End: 
