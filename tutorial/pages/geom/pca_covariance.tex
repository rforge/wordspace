\begin{frame}
  \frametitle{The covariance matrix}
  %% \framesubtitle{}
  
  \begin{itemize}
  \item Find the direction $\vv$ with maximal $\sigma_{\vv}^2$, which is given by:
  \end{itemize}
  \ungap
  \begin{align*}
    \sigma_{\vv}^2 &= \tfrac{1}{k-1} \sum_{i=1}^k \sprod{\vx_i}{\vv}^2 
    \\
    \only<beamer:2-| handout:1>{
      &= \tfrac{1}{k-1} \sum_{i=1}^k \left(\vx_i^T \vv\right)^T
      \cdot \left(\vx_i^T \vv\right) }
    \\
    \only<beamer:3-| handout:1>{
      &= \tfrac{1}{k-1} \sum_{i=1}^k \vv^T  
      \left(\vx_i  \vx_i^T \right) \vv }
    \\
    \only<beamer:4-| handout:1>{
      &= \vv^T  
      \left( \tfrac{1}{k-1} \sum_{i=1}^k \vx_i \vx_i^T \right) 
      \vv} 
    \\
    \only<beamer:5-| handout:1>{
      &= \vv^T \mathbf{C} \vv}
  \end{align*}
\end{frame}

\begin{frame}
  \frametitle{The covariance matrix}
  %% \framesubtitle{}
  
  \begin{itemize}
  \item $\mathbf{C}$ is the \h{covariance matrix} of the data points
    \begin{itemize}
    \item $\mathbf{C}$ is a square $n\times n$ matrix ($2\times 2$ in our
      example)
    \end{itemize}
  \item Preserved variance after projection onto a line $\vv$ can easily be
    calculated as $\sigma_{\vv}^2 = \vv^T \mathbf{C} \vv$
  \item<2-> The original variance of the data set is given by $\sigma^2 =
    \mathop{\text{tr}}(\mathbf{C}) = C_{11} + C_{22} + \cdots + C_{nn}$
    \[
    \mathbf{C} =
    \begin{bmatrix}
      \primary{\sigma_1^2} & C_{12} & \cdots & C_{1n} \\
      C_{21} & \primary{\sigma_2^2} & \ddots & \vdots \\
      \\
      \vdots & \ddots & \primary{\ddots} & C_{n-1,n} \\
      \\
      C_{n1} & \cdots & C_{n,n-1}& \primary{\sigma_n^2}
    \end{bmatrix}
    \]
  \end{itemize}
\end{frame}

%%% Local Variables: 
%%% mode: latex
%%% TeX-master: "../../workspace"
%%% End: 
