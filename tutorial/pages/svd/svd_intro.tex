\begin{frame}
  \frametitle{Remember PCA?}
  %% \framesubtitle{}

  \begin{itemize}
  \item Principal components analysis is based on an \h{eigenvalue
      decomposition} of the \h{covariance matrix} $\mathbf{C}$ into
    \[
    \mathbf{C} = \mathbf{U}\cdot \mathbf{D}\cdot \mathbf{U}^T
    \]
    where $\mathbf{U}$ is orthogonal and $\mathbf{D} =
    \mathop{\text{Diag}}(\lambda_1, \ldots, \lambda_n)$.
  \item<2-> The columns of $\mathbf{U}$ are \hh{eigenvectors}
    \begin{center}
      $\mathbf{C} \va_i = \lambda_i \va_i$
    \end{center}
    for the ordered \hh{eigenvalues} $\lambda_1\geq \lambda_2\geq \dots \geq
    \lambda_n$
    \gap[.5]
  \item<3-> Interesting link: $\vu^T \mathbf{C} \vv$ describes a general
    \h{inner product}
    \begin{itemize}
    \item $\sigma_{\vv}$ is the norm of $\vv$ with respect to this general
      inner product
    \item the eigenvalue decomposition corresponds to a transformation into
      Cartesian coordinates where $\mathbf{C}$ has diagonal form
    \item eigenvalues $\lambda_i$ are the ``squashing factors'' of the unit
      circle
    \end{itemize}
  \end{itemize}
\end{frame}

\begin{frame}
  \frametitle{Singular value decomposition (SVD)}
  %% \framesubtitle{}

  \begin{itemize}
  \item The idea of eigenvalue decomposition can be generalised to an
    arbitrary (non-symmetric, non-square) matrix $\mathbf{A}$
    \begin{itemize}
    \item[\hand] need not have any eigenvalues
    \end{itemize}
  \item<2-> \h{Singular value decomposition} (\hh{SVD}) factorises $\mathbf{A}$ into
    \[
    \mathbf{A} = \mathbf{U}\cdot \Msigma\cdot \mathbf{V}^T
    \]
    where $\mathbf{U}$ and $\mathbf{V}$ are orthogonal coordinate
    transformations and $\Msigma$ is a rectangular-diagonal matrix of
    \hh{singular values}\\
    (with customary ordering $\sigma_1\geq \sigma_2\geq \dots \geq
    \sigma_n\geq 0$)
  \item<3-> SVD is an important tool in linear algebra and statistics
    \begin{itemize}
    \item[\hand] in particular, PCA can be computed from SVD decomposition
    \end{itemize}
  \end{itemize}
\end{frame}

\begin{frame}[c]
  \frametitle{SVD illustration}
  %% \framesubtitle{}
  
  \begin{equation*}
    \begin{bmatrix}
      & & \primary{n} & & \\
      & & & & \\
      & & & & \\
      \primary{k} & & \mathbf{A} & & \\
      & & & & \\
      & & & & \\
      & & & & 
    \end{bmatrix}
    =
    \begin{bmatrix}
      & & & \primary{k} & & & \\
      & & & & & & \\
      & & & & & & \\
      \primary{k} & & & \mathbf{U} & & & \\
      & & & & & & \\
      & & & & & & \\
      & & & & & &
    \end{bmatrix}
    \cdot
    \begin{bmatrix}
      \sigma_1 & \primary{n} & \\
      & \ddots & \\
      & & \sigma_n \\
      \primary{k} & \Msigma & \\
      & & \\
      & & \rule{0mm}{5mm} 
    \end{bmatrix}
    \cdot
    \begin{bmatrix}
      & & \primary{n} & & \\
      & & & & \\
      \primary{n} & & \mathbf{V}^T & & \\
      & & & & \\
      & & & &
    \end{bmatrix}
  \end{equation*}
\end{frame}

%%% Local Variables: 
%%% mode: latex
%%% TeX-master: "../../workspace"
%%% End: 
